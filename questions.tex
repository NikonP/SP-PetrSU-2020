\documentclass{article}
\usepackage{xcolor}
\usepackage{listings}
\usepackage[english, russian]{babel}
\usepackage{graphicx}
\graphicspath{ {./images/} }
\usepackage{cancel}
\usepackage{hyperref}
\usepackage{geometry}

\hypersetup{
    hyperfootnotes=false,
    colorlinks=true,
    linkcolor = black,
    filecolor=magenta,      
    urlcolor=cyan
}

\footskip=50pt
\geometry{bmargin = 80pt}

\definecolor{mGreen}{rgb}{0,0.6,0}
\definecolor{mGray}{rgb}{0.5,0.5,0.5}
\definecolor{mPurple}{rgb}{0.58,0,0.82}
\definecolor{backgroundColour}{rgb}{0.95,0.95,0.92}

\lstdefinestyle{CStyle}{
    backgroundcolor=\color{backgroundColour},   
    commentstyle=\color{mGreen},
    keywordstyle=\color{magenta},
    numberstyle=\tiny\color{mGray},
    stringstyle=\color{mPurple},
    basicstyle=\footnotesize,
    breakatwhitespace=false,         
    breaklines=true,                 
    captionpos=b,                    
    keepspaces=true,                 
    numbers=left,                    
    numbersep=5pt,                  
    showspaces=false,                
    showstringspaces=false,
    showtabs=false,                  
    tabsize=2,
    language=C
}

\begin{document}

\vspace*{\fill}
\tableofcontents
\vspace*{\fill}
\newpage
% \begin{lstlisting}[style=CStyle]
% #include <stdio.h>
% int main(int argc, char* argv[])
% {
%    printf("Hello World!"); 
% }
% \end{lstlisting}

\section{Обзор ОС UNIX: архитектура, вход в систему, файлы и каталоги, ввод и вывод.}
\section{Обзор ОС UNIX: программы и процессы, обработка ошибок, идентификация пользователя.}
\section{Обзор ОС UNIX: сигналы, представление времени, системные вызовы и библиотечные функции.}
\section{Стандарты и реализации ОС UNIX: пределы ISO C, пределы POSIX, функции sysconf(), pathconf() и fpathconf(), элементарные системные типы данных.}
\section{Файловый ввод-вывод: дескрипторы файлов, функция open(), функция creat(), функция close().}
\section{Файловый ввод-вывод: Функция lseek(), функция read(), функция write()/}
\section{Файловый ввод-вывод: эффективность операций ввода-вывода}
\section{Файловый ввод-вывод: совместное использование файлов, атомарные операции, функции dup() и dup2()}
\section{Файловый ввод-вывод: функции sync(), fsync(), fdatasync(), fcntl(), ioctl(), /dev/fd}
\section{Файлы и каталоги: функции stat(), fstat(), lstat(), содержимое struct stat.}
\section{Файлы и каталоги: типы файлов, права доступа к файлу, функция umask().}
\section{Файлы и каталоги: функции chmod(), fchmod(), chown(), fchown(), lchown().}
\section{Файлы и каталоги: размер файла, дырки в файлах, усечение файлов, файловые системы, функции link(), unlink(), remove(), rename().}
\section{Файлы и каталоги: символические ссылки, функции symlink() и readlink().}
\newpage
\section{Файлы и каталоги: временные характеристики файлов, функция utime().}
\section{Файлы и каталоги: функции mkdir() и rmdir(), чтение каталогов, функции chdir(), fchdir(), getcwd().}
\section{Стандартная библиотека ввода-вывода: потоки и объекты FILE, стандартные потоки ввода, вывода и сообщений об ошибках, буферизация.}
\section{Стандартная библиотека ввода-вывода: открытие потока, чтение из потока и запись в поток, функции ввода, функции вывода.}
\section{Стандартная библиотека ввода-вывода: эффективность стандартных операций ввода-вывода, позиционирование в потоке.}
\section{Стандартная библиотека ввода-вывода: форматированный вывод, форматированный ввод, временные файлы.}
\section{Управление процессами: идентификаторы процесса, функция fork(), совместное использование файлов.}
\section{Управление процессами: функция exit(), функции wait() и waitpid().}
\section{Управление процессами: семейство функций exec().}
\section{Управление процессами: изменение идентификаторов пользователя и группы, функции setuid(), setgid(), seteuid(), setegid().}
\section{Управление процессами: интерпретируемые файлы, функция system().}
\section{Сигналы: концепция сигналов, функция signal(), ненадежные сигналы.}
\section{Сигналы: прерванные системные вызовы, реентерабельные функции.}
\section{Сигналы: функции kill(), raise(), alarm(), pause().}
\section{Сигналы: надежные сигналы, терминология и семантика, наборы сигналов.}
\section{Сигналы: маска сигналов процесса и функция sigprocmask(), функция sigpending(),}
\section{Сигналы: функция sigaction().}
\section{Сигналы: функция sigsuspend().}
\end{document} 